%% Generated by Sphinx.
\def\sphinxdocclass{report}
\documentclass[letterpaper,10pt,english]{sphinxmanual}
\ifdefined\pdfpxdimen
   \let\sphinxpxdimen\pdfpxdimen\else\newdimen\sphinxpxdimen
\fi \sphinxpxdimen=.75bp\relax

\PassOptionsToPackage{warn}{textcomp}
\usepackage[utf8]{inputenc}
\ifdefined\DeclareUnicodeCharacter
% support both utf8 and utf8x syntaxes
  \ifdefined\DeclareUnicodeCharacterAsOptional
    \def\sphinxDUC#1{\DeclareUnicodeCharacter{"#1}}
  \else
    \let\sphinxDUC\DeclareUnicodeCharacter
  \fi
  \sphinxDUC{00A0}{\nobreakspace}
  \sphinxDUC{2500}{\sphinxunichar{2500}}
  \sphinxDUC{2502}{\sphinxunichar{2502}}
  \sphinxDUC{2514}{\sphinxunichar{2514}}
  \sphinxDUC{251C}{\sphinxunichar{251C}}
  \sphinxDUC{2572}{\textbackslash}
\fi
\usepackage{cmap}
\usepackage[T1]{fontenc}
\usepackage{amsmath,amssymb,amstext}
\usepackage{babel}



\usepackage{times}
\expandafter\ifx\csname T@LGR\endcsname\relax
\else
% LGR was declared as font encoding
  \substitutefont{LGR}{\rmdefault}{cmr}
  \substitutefont{LGR}{\sfdefault}{cmss}
  \substitutefont{LGR}{\ttdefault}{cmtt}
\fi
\expandafter\ifx\csname T@X2\endcsname\relax
  \expandafter\ifx\csname T@T2A\endcsname\relax
  \else
  % T2A was declared as font encoding
    \substitutefont{T2A}{\rmdefault}{cmr}
    \substitutefont{T2A}{\sfdefault}{cmss}
    \substitutefont{T2A}{\ttdefault}{cmtt}
  \fi
\else
% X2 was declared as font encoding
  \substitutefont{X2}{\rmdefault}{cmr}
  \substitutefont{X2}{\sfdefault}{cmss}
  \substitutefont{X2}{\ttdefault}{cmtt}
\fi


\usepackage[Bjarne]{fncychap}
\usepackage{sphinx}

\fvset{fontsize=\small}
\usepackage{geometry}

% Include hyperref last.
\usepackage{hyperref}
% Fix anchor placement for figures with captions.
\usepackage{hypcap}% it must be loaded after hyperref.
% Set up styles of URL: it should be placed after hyperref.
\urlstyle{same}

\usepackage{sphinxmessages}




\title{Flood Tool}
\date{Oct 24, 2019}
\release{}
\author{unknown}
\newcommand{\sphinxlogo}{\vbox{}}
\renewcommand{\releasename}{}
\makeindex
\begin{document}

\pagestyle{empty}
\sphinxmaketitle
\pagestyle{plain}
\sphinxtableofcontents
\pagestyle{normal}
\phantomsection\label{\detokenize{index::doc}}


This package implements a flood risk prediction tool.


\chapter{Geodetic Transformations}
\label{\detokenize{index:geodetic-transformations}}
For historical reasons, multiple coordinate systems exist in British mapping.
The Ordnance Survey has been mapping the British Isles since the 18th Century
and the last major retriangulation from 1936-1962 produced the Ordance Survey
National Grid (or \sphinxstylestrong{OSGB36}), which defined latitude and longitude across the
island of Great Britain %
\begin{footnote}[1]\sphinxAtStartFootnote
A guide to coordinate systems in Great Britain, Ordnance Survey
%
\end{footnote}. For convenience, a standard Transverse Mercator
projection %
\begin{footnote}[2]\sphinxAtStartFootnote
Map projections - A Working Manual, John P. Snyder, \sphinxurl{https://doi.org/10.3133/pp1395}
%
\end{footnote} was also defined, producing a notionally flat gridded surface,
with gradations called eastings and westings. The scale for these gradations
was identified with metres.

The OSGB36 datum is based on the Airy Ellipsoid of 1830, which defines
semimajor axes for its model of the earth, \(a\) and \(b\), a scaling
factor \(F_0\) and ellipsoid height, \(H\).
\begin{equation*}
\begin{split}a &= 6377563.396, \\
b &= 6356256.910, \\
F_0 &= 0.9996012717, \\
H &= 24.7.\end{split}
\end{equation*}
The point of origin for the transverse Mercator projection is defined in the
Ordnance Survey longitude-latitude and easting-northing coordinates as
\begin{equation*}
\begin{split}\phi^{OS}_0 &= 49^\circ \mbox{ north}, \\
\lambda^{OS}_0 &= 2^\circ \mbox{ west}, \\
E^{OS}_0 &= 400000 m, \\
N^{OS}_0 &= -100000 m.\end{split}
\end{equation*}
More recently, the world has gravitated towards the use of Satellite based GPS
equipment, which uses the (globally more appropriate) World Geodetic System
1984 (or \sphinxstylestrong{WGS84}). This datum uses a different ellipsoid, which offers a
better fit for a global coordinate system. Its key properties are:
\begin{equation*}
\begin{split}a_{WGS} &= 6378137,, \\
b_{WGS} &= 6356752.314, \\
F_0 &= 0.9996.\end{split}
\end{equation*}
For a given point on the WGS84 ellipsoid, an approximate mapping to the
OSGB36 datum can be found using a Helmert transformation %
\begin{footnote}[3]\sphinxAtStartFootnote
Computing Helmert transformations, G Watson, \sphinxurl{http://www.maths.dundee.ac.uk/gawatson/helmertrev.pdf}
%
\end{footnote},
\begin{equation*}
\begin{split}\mathbf{x}^{OS} = \mathbf{t}+\mathbf{M}\mathbf{x}^{WGS}.\end{split}
\end{equation*}
Here \(\mathbf{x}\) denotes a coordinate in Cartesian space (i.e in 3D)
as given by the (invertible) transformation
\begin{equation*}
\begin{split}\nu &= \frac{aF_0}{\sqrt{1-e^2\sin^2(\phi^{OS})}} \\
x &= (\nu+H) \sin(\lambda)\cos(\phi) \\
y &= (\nu+H) \cos(\lambda)\cos(\phi) \\
z &= ((1-e^2)\nu+H)\sin(\phi)\end{split}
\end{equation*}
and the transformation parameters are
\begin{eqnarray*}
\mathbf{t} &= \left(\begin{array}{c}
-446.448\\ 125.157\\ -542.060
\end{array}\right),\\
\mathbf{M} &= \left[\begin{array}{ c c c }
1+s& -r_3& r_2\\
r_3 & 1+s & -r_1 \\
-r_2 & r_1 & 1+s
\end{array}\right], \\
s &= 20.4894\times 10^{-6}, \\
\mathbf{r} &= [0.1502'', 0.2470'', 0.8421''].
\end{eqnarray*}
Given a latitude, \(\phi^{OS}\) and longitude, \(\lambda^{OS}\) in the
OSGB36 datum, easting and northing coordinates, \(E^{OS}\) \& \(N^{OS}\)
can then be calculated using the following formulae:
\begin{equation*}
\begin{split}\rho &= \frac{aF_0(1-e^2)}{\left(1-e^2\sin^2(\phi^{OS})\right)^{\frac{3}{2}}} \\
\eta &= \sqrt{\frac{\nu}{\rho}-1} \\
M &= bF_0\left[\left(1+n+\frac{5}{4}n^2+\frac{5}{4}n^3\right)(\phi^{OS}-\phi^{OS}_0)\right. \\
&\quad-\left(3n+3n^2+\frac{21}{8}n^3\right)\sin(\phi-\phi_0)\cos(\phi^{OS}+\phi^{OS}_0) \\
&\quad+\left(\frac{15}{8}n^2+\frac{15}{8}n^3\right)\sin(2(\phi^{OS}-\phi^{OS}_0))\cos(2(\phi^{OS}+\phi^{OS}_0)) \\
&\left.\quad-\frac{35}{24}n^3\sin(3(\phi-\phi_0))\cos(3(\phi^{OS}+\phi^{OS}_0))\right] \\
I &= M + N^{OS}_0 \\
II &= \frac{\nu}{2}\sin(\phi^{OS})\cos(\phi^{OS}) \\
III &= \frac{\nu}{24}\sin(\phi^{OS})cos^3(\phi^{OS})(5-\tan^2(phi^{OS})+9\eta^2) \\
IIIA &= \frac{\nu}{720}\sin(\phi^{OS})cos^5(\phi^{OS})(61-58\tan^2(\phi^{OS})+\tan^4(\phi^{OS})) \\
IV &= \nu\cos(\phi^{OS}) \\
V &= \frac{\nu}{6}\cos^3(\phi^{OS})\left(\frac{\nu}{\rho}-\tan^2(\phi^{OS})\right) \\
VI &= \frac{\nu}{120}\cos^5(\phi^{OS})(5-18\tan^2(\phi^{OS})+\tan^4(\phi^{OS}) \\
&\quad+14\eta^2-58\tan^2(\phi^{OS})\eta^2) \\
E^{OS} &= E^{OS}_0+IV(\lambda^{OS}-\lambda^{OS}_0)+V(\lambda-\lambda^{OS}_0)^3+VI(\lambda^{OS}-\lambda^{OS}_0)^5 \\
N^{OS} &= I + II(\lambda^{OS}-\lambda^{OS}_0)^2+III(\lambda-\lambda^{OS}_0)^4+IIIA(\lambda^{OS}-\lambda^{OS}_0)^6\end{split}
\end{equation*}

\chapter{Function APIs}
\label{\detokenize{index:module-flood_tool}}\label{\detokenize{index:function-apis}}\index{flood\_tool (module)@\spxentry{flood\_tool}\spxextra{module}}\index{Tool (class in flood\_tool)@\spxentry{Tool}\spxextra{class in flood\_tool}}

\begin{fulllineitems}
\phantomsection\label{\detokenize{index:flood_tool.Tool}}\pysiglinewithargsret{\sphinxbfcode{\sphinxupquote{class }}\sphinxcode{\sphinxupquote{flood\_tool.}}\sphinxbfcode{\sphinxupquote{Tool}}}{\emph{postcode\_file=None}, \emph{risk\_file=None}, \emph{values\_file=None}}{}
Class to interact with a postcode database file.

Reads postcode and flood risk files and provides a postcode locator service.
\begin{quote}\begin{description}
\item[{Parameters}] \leavevmode\begin{itemize}
\item {} 
\sphinxstyleliteralstrong{\sphinxupquote{postcode\_file}} (\sphinxstyleliteralemphasis{\sphinxupquote{str}}\sphinxstyleliteralemphasis{\sphinxupquote{, }}\sphinxstyleliteralemphasis{\sphinxupquote{optional}}) \textendash{} Filename of a .csv file containing geographic location data for postcodes.

\item {} 
\sphinxstyleliteralstrong{\sphinxupquote{risk\_file}} (\sphinxstyleliteralemphasis{\sphinxupquote{str}}\sphinxstyleliteralemphasis{\sphinxupquote{, }}\sphinxstyleliteralemphasis{\sphinxupquote{optional}}) \textendash{} Filename of a .csv file containing flood risk data.

\item {} 
\sphinxstyleliteralstrong{\sphinxupquote{postcode\_file}} \textendash{} Filename of a .csv file containing property value data for postcodes.

\end{itemize}

\end{description}\end{quote}
\index{get\_annual\_flood\_risk() (flood\_tool.Tool method)@\spxentry{get\_annual\_flood\_risk()}\spxextra{flood\_tool.Tool method}}

\begin{fulllineitems}
\phantomsection\label{\detokenize{index:flood_tool.Tool.get_annual_flood_risk}}\pysiglinewithargsret{\sphinxbfcode{\sphinxupquote{get\_annual\_flood\_risk}}}{\emph{postcodes}, \emph{probability\_bands}}{}
Get an array of estimated annual flood risk in pounds sterling per year of a flood
event from a sequence of postcodes and flood probabilities.
\begin{quote}\begin{description}
\item[{Parameters}] \leavevmode\begin{itemize}
\item {} 
\sphinxstyleliteralstrong{\sphinxupquote{postcodes}} (\sphinxstyleliteralemphasis{\sphinxupquote{sequence of strs}}) \textendash{} Ordered collection of postcodes

\item {} 
\sphinxstyleliteralstrong{\sphinxupquote{probability\_bands}} (\sphinxstyleliteralemphasis{\sphinxupquote{sequence of strs}}) \textendash{} Ordered collection of flood probabilities

\end{itemize}

\item[{Returns}] \leavevmode
array of floats for the annual flood risk in pounds sterling for the input postcodes.
Invalid postcodes return \sphinxtitleref{numpy.nan}.

\item[{Return type}] \leavevmode
numpy.ndarray

\end{description}\end{quote}

\end{fulllineitems}

\index{get\_easting\_northing\_flood\_probability() (flood\_tool.Tool method)@\spxentry{get\_easting\_northing\_flood\_probability()}\spxextra{flood\_tool.Tool method}}

\begin{fulllineitems}
\phantomsection\label{\detokenize{index:flood_tool.Tool.get_easting_northing_flood_probability}}\pysiglinewithargsret{\sphinxbfcode{\sphinxupquote{get\_easting\_northing\_flood\_probability}}}{\emph{easting}, \emph{northing}}{}
Get an array of flood risk probabilities from arrays of eastings and northings.

Flood risk data is extracted from the Tool flood risk file. Locations
not in a risk band circle return \sphinxtitleref{Zero}, otherwise returns the name of the
highest band it sits in.
\begin{quote}\begin{description}
\item[{Parameters}] \leavevmode\begin{itemize}
\item {} 
\sphinxstyleliteralstrong{\sphinxupquote{easting}} (\sphinxstyleliteralemphasis{\sphinxupquote{numpy.ndarray of floats}}) \textendash{} OS Eastings of locations of interest

\item {} 
\sphinxstyleliteralstrong{\sphinxupquote{northing}} (\sphinxstyleliteralemphasis{\sphinxupquote{numpy.ndarray of floats}}) \textendash{} Ordered sequence of postcodes

\end{itemize}

\item[{Returns}] \leavevmode
numpy array of flood probability bands corresponding to input locations.

\item[{Return type}] \leavevmode
numpy.ndarray of strs

\end{description}\end{quote}

\end{fulllineitems}

\index{get\_flood\_cost() (flood\_tool.Tool method)@\spxentry{get\_flood\_cost()}\spxextra{flood\_tool.Tool method}}

\begin{fulllineitems}
\phantomsection\label{\detokenize{index:flood_tool.Tool.get_flood_cost}}\pysiglinewithargsret{\sphinxbfcode{\sphinxupquote{get\_flood\_cost}}}{\emph{postcodes}}{}
Get an array of estimated cost of a flood event from a sequence of postcodes.
:param postcodes: Ordered collection of postcodes
:type postcodes: sequence of strs
:param probability\_bands: Ordered collection of flood probability bands
:type probability\_bands: sequence of strs
\begin{quote}\begin{description}
\item[{Returns}] \leavevmode
array of floats for the pound sterling cost for the input postcodes.
Invalid postcodes return \sphinxtitleref{numpy.nan}.

\item[{Return type}] \leavevmode
numpy.ndarray of floats

\end{description}\end{quote}

\end{fulllineitems}

\index{get\_lat\_long() (flood\_tool.Tool method)@\spxentry{get\_lat\_long()}\spxextra{flood\_tool.Tool method}}

\begin{fulllineitems}
\phantomsection\label{\detokenize{index:flood_tool.Tool.get_lat_long}}\pysiglinewithargsret{\sphinxbfcode{\sphinxupquote{get\_lat\_long}}}{\emph{postcodes}}{}
Get an array of WGS84 (latitude, longitude) pairs from a list of postcodes.
\begin{quote}\begin{description}
\item[{Parameters}] \leavevmode
\sphinxstyleliteralstrong{\sphinxupquote{postcodes}} (\sphinxstyleliteralemphasis{\sphinxupquote{sequence of strs}}) \textendash{} Ordered sequence of N postcode strings

\item[{Returns}] \leavevmode
Array of Nx2 (latitude, longitdue) pairs for the input postcodes.
Invalid postcodes return {[}\sphinxtitleref{numpy.nan}, \sphinxtitleref{numpy.nan}{]}.

\item[{Return type}] \leavevmode
ndarray

\end{description}\end{quote}

\end{fulllineitems}

\index{get\_sorted\_annual\_flood\_risk() (flood\_tool.Tool method)@\spxentry{get\_sorted\_annual\_flood\_risk()}\spxextra{flood\_tool.Tool method}}

\begin{fulllineitems}
\phantomsection\label{\detokenize{index:flood_tool.Tool.get_sorted_annual_flood_risk}}\pysiglinewithargsret{\sphinxbfcode{\sphinxupquote{get\_sorted\_annual\_flood\_risk}}}{\emph{postcodes}}{}
Get a sorted pandas DataFrame of flood risks.
\begin{quote}\begin{description}
\item[{Parameters}] \leavevmode
\sphinxstyleliteralstrong{\sphinxupquote{postcodes}} (\sphinxstyleliteralemphasis{\sphinxupquote{sequence of strs}}) \textendash{} Ordered sequence of postcodes

\item[{Returns}] \leavevmode
Dataframe of flood risks indexed by (normalized) postcode and ordered by risk,
then by lexagraphic (dictionary) order on the postcode. The index is named
\sphinxtitleref{Postcode} and the data column \sphinxtitleref{Flood Risk}.
Invalid postcodes and duplicates are removed.

\item[{Return type}] \leavevmode
pandas.DataFrame

\end{description}\end{quote}

\end{fulllineitems}

\index{get\_sorted\_flood\_probability() (flood\_tool.Tool method)@\spxentry{get\_sorted\_flood\_probability()}\spxextra{flood\_tool.Tool method}}

\begin{fulllineitems}
\phantomsection\label{\detokenize{index:flood_tool.Tool.get_sorted_flood_probability}}\pysiglinewithargsret{\sphinxbfcode{\sphinxupquote{get\_sorted\_flood\_probability}}}{\emph{postcodes}}{}
Get an array of flood risk probabilities from a sequence of postcodes.

Probability is ordered High\textgreater{}Medium\textgreater{}Low\textgreater{}Very low\textgreater{}Zero.
Flood risk data is extracted from the \sphinxtitleref{Tool} flood risk file.
\begin{quote}\begin{description}
\item[{Parameters}] \leavevmode
\sphinxstyleliteralstrong{\sphinxupquote{postcodes}} (\sphinxstyleliteralemphasis{\sphinxupquote{sequence of strs}}) \textendash{} Ordered sequence of postcodes

\item[{Returns}] \leavevmode
Dataframe of flood probabilities indexed by postcode and ordered from \sphinxtitleref{High} to \sphinxtitleref{Zero},
then by lexagraphic (dictionary) order on postcode. The index is named \sphinxtitleref{Postcode}, the
data column is named \sphinxtitleref{Probability Band}. Invalid postcodes and duplicates
are removed.

\item[{Return type}] \leavevmode
pandas.DataFrame

\end{description}\end{quote}

\end{fulllineitems}


\end{fulllineitems}

\index{WGS84toOSGB36() (in module flood\_tool)@\spxentry{WGS84toOSGB36()}\spxextra{in module flood\_tool}}

\begin{fulllineitems}
\phantomsection\label{\detokenize{index:flood_tool.WGS84toOSGB36}}\pysiglinewithargsret{\sphinxcode{\sphinxupquote{flood\_tool.}}\sphinxbfcode{\sphinxupquote{WGS84toOSGB36}}}{\emph{latitude}, \emph{longitude}, \emph{radians=False}}{}
Wrapper to transform (latitude, longitude) pairs
from GPS to OS datum.

\end{fulllineitems}

\index{get\_easting\_northing\_from\_lat\_long() (in module flood\_tool)@\spxentry{get\_easting\_northing\_from\_lat\_long()}\spxextra{in module flood\_tool}}

\begin{fulllineitems}
\phantomsection\label{\detokenize{index:flood_tool.get_easting_northing_from_lat_long}}\pysiglinewithargsret{\sphinxcode{\sphinxupquote{flood\_tool.}}\sphinxbfcode{\sphinxupquote{get\_easting\_northing\_from\_lat\_long}}}{\emph{latitude}, \emph{longitude}, \emph{radians=False}}{}
Convert GPS (latitude, longitude) to OS (easting, northing).
\begin{quote}\begin{description}
\item[{Parameters}] \leavevmode\begin{itemize}
\item {} 
\sphinxstyleliteralstrong{\sphinxupquote{latitude}} (\sphinxstyleliteralemphasis{\sphinxupquote{sequence of floats}}) \textendash{} Latitudes to convert.

\item {} 
\sphinxstyleliteralstrong{\sphinxupquote{longitude}} (\sphinxstyleliteralemphasis{\sphinxupquote{sequence of floats}}) \textendash{} Lonitudes to convert.

\item {} 
\sphinxstyleliteralstrong{\sphinxupquote{radians}} (\sphinxstyleliteralemphasis{\sphinxupquote{bool}}\sphinxstyleliteralemphasis{\sphinxupquote{, }}\sphinxstyleliteralemphasis{\sphinxupquote{optional}}) \textendash{} Set to \sphinxtitleref{True} if input is in radians. Otherwise degrees are assumed

\end{itemize}

\item[{Returns}] \leavevmode
\begin{itemize}
\item {} 
\sphinxstylestrong{easting} (\sphinxstyleemphasis{ndarray of floats}) \textendash{} OS Eastings of input

\item {} 
\sphinxstylestrong{northing} (\sphinxstyleemphasis{ndarray of floats}) \textendash{} OS Northings of input

\end{itemize}


\end{description}\end{quote}
\subsubsection*{References}

A guide to coordinate systems in Great Britain
(\sphinxurl{https://webarchive.nationalarchives.gov.uk/20081023180830/http://www.ordnancesurvey.co.uk/oswebsite/gps/information/coordinatesystemsinfo/guidecontents/index.html})

\end{fulllineitems}

\subsubsection*{References}


\renewcommand{\indexname}{Python Module Index}
\begin{sphinxtheindex}
\let\bigletter\sphinxstyleindexlettergroup
\bigletter{f}
\item\relax\sphinxstyleindexentry{flood\_tool}\sphinxstyleindexpageref{index:\detokenize{module-flood_tool}}
\end{sphinxtheindex}

\renewcommand{\indexname}{Index}
\printindex
\end{document}